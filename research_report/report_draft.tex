\documentclass[11pt]{article}
%\twocolumn


\usepackage{graphicx}
\usepackage[justification=centering]{caption}
\graphicspath{Figss/}
\usepackage{amsmath,amssymb}
\usepackage{a4wide}
\usepackage{physics}
\usepackage{multicol}
\usepackage{multirow}
\usepackage[round]{natbib}
\def\ni{\noindent}
\begin{document}

\title{\bf{Network Inference in Encephalography Time Series using Partial Correlation}}
\author{Rick Evertz 9548866\\
Supervisor: Prof. Damien Hicks\\
Swinburne University of Technology}
\date{}
\maketitle
%\abstract 
%\ni Investigating brain networks of people in conscious and unconscious states.
%\pagebreak

%\tableofcontents
%\pagebreak
%\begin{document}

\section{Introduction}
Our ability to interpret and understand the Universe, is ultimately dependent on our brain and its interactions with the outside world. Everything we see, feel and do, the diverse spectrum of human experience, is all filtered through the lens of our conscious and sub-conscious experience. The nervous system receives and transports a complicated series of stimulus to the brain, which in turn interprets these signals and builds up a picture of our surroundings. Without subscribing to supernatural causes or otherworldy origins, consciousness is then, almost certainly an emergent property of the complex interactions occurring within the brain. Therefore, it is crucial to our understanding of the self and by extension, the cosmos, that we develop a sound method to study the inner workings of the brain. Several clever methods have been developed to investigate the structural and behavioral properties of the mind. The observation that the brain at an abstract level can be described as a complex network has led to an expansion of our understanding of the brain and how its constituents components interact. This research thesis aims to learn the functional network structure present in neural activity data. To be able to draw sound conclusions from any network analysis of the brain, we must be sure that the tools we use for inference are well founded and tested extensively, thus the desire for a robust and justifiable structure learning method emerges. The aim of this research thesis is to develop a network inference method that can be applied to magnetoencephalography (MEG) and electroencephalography (EEG) time series. Research is motivated by work completed in the brain network community, where it has become increasingly relevant to have a robust and justifiable method for learning networks allowing the construction of functional connectivity networks. We aim to build functional connectivity networks out of the time series data by inferring pairwise interactions using partial correlation. This approach is model and assumption free, with partial correlation chosen as the most suitable network inference tool for our purposes.

\section{Literature Review}
It has become increasingly clear that there is a deep need for a robust and empirically justifiable method for inferring functional networks in neural activity recordings. Many techniques exists currently for learning network structure, but few, if any have been rigorously tested prior to application. The situation is further complicated by the fact that many of these networks are being used to make clinical and physiological statements about the brain. Much of the functional connectivity work done in neuroscience has involved source localization in one form or another. Source localization is the inference of the physical origin of the measured electric potentials or magnetic fields collected using EEG and MEG, within the brain volume \citep{friston}. We suggest that rather source localizing using methods that have not been extensively verified, it is more sensible to to learn the network at the sensor level, using only the measured times series data, with no heuristic assumptions made prior. 

\subsection{General Background on Network Theory}
Networks are a map of the relations between objects, they are constructed using nodes and edges. The mathematical description of networks is founded in graph theory. Nodes represent the objects in the network and edges are the connections between them. A simple example is a social network, with people representing nodes and the relationships between them, the edges. Networks can be small and simple, or they can be vast and complex in nature, and everything in between. The study of complex networks is as diverse as it is fascinating, with applications ranging across many fields. A brief survey of complex networks research presents general trends with researchers investigating the network structure and evolution. It is desirable to be able to condense a network down into a set of quantitative statements by application of a graph metric. A simple example is that of the node degree, consider a social network, it may be of interest to know who is the most popular? This could easily be achieved by summing the number of edges each node has (assuming popularity is based on the number of friends). Networks exist everywhere in our daily lives. From the public transport people use on their daily commute, to the constant connections we have with our electronic devices to the Internet, we have never been more integrated into such a diverse range of networks. Interestingly, many of these real world networks present similar properties in their topology as described by \citet{albert2002statistical} in \emph{Statistical mechanics of complex networks}. Perhaps it is no surprise then that the study of complex networks has been applied to the brain. The diverse range of phenomena that can be studied by modeling as a complex network has led naturally to investigations of brain structure. The brain is a vast and complicated network consisting of billions of neurons and trillions of connections. The construction of brain networks, be that, the scale and how nodes and edges are defined, has many avenues.
\begin{figure}[h!]%[H]
\centering\includegraphics[scale = 0.7]{brain1.png}%width=\linewidth
\caption{General procedure for building brain networks -  from \citep{fall}}
\label{b1}
\end{figure}

\subsection{Learning the network}
Construction of functional connectivity networks includes a series of complicated steps. Figure \ref{b1} is a pictorial representation of the general procedure followed when constructing brain networks. The process begins with the collection of neural activity data, which is either performed in an non-invasive manner e.g. fMRI, MEG and EEG, or less desirably, through invasive means via electrodes implanted directly into the cerebral matter \citep{friston}. The process of learning the network begins with the neural data being segmented into a series of nodes. The choice of node definition will depend on the type of data and research aims \citep{stnode}. With the nodes of the network defined, the process of inferring either structural or functional links begins, with the final result producing a connectivity matrix, which is a matrix listing the pairwise interactions between nodes. The next stage involves thresholding the connectivity matrix to obtain either a weight or binary adjacency matrix. No current standard has yet been set for thresholding connectivity matrices and is an area of ongoing research \citep{papo}. Post thresholding, the adjacency matrix, which provides the pairwise relations between nodes, is analyzed using graph metrics. The application of the graph metrics will then provide insight into the topology of the network or functional changes over time. The final step, is only performed in studies that are interested in investigating the properties of brain networks in illness and health. Researchers pose a question and compare the network properties of those obtained from brains suffering from disease and those of healthy volunteers in attempt to understand illness in terms of network structure \citep{fall}

\subsection{General Brain Networks Studies (Structural)}
Research on the properties and dynamics of the human brain has expanded significantly with the application of complex network analysis. Network theory has many applications in the brain, with the desire to be able to derive quantitative information on brain physiology and function. Brain connectivity networks are separated into three broad classes; structural, functional and effective connectivity measures \citep{fornito2013graph}. Structural connectivity networks are built using anatomical connections between brain regions. Functional connectivity is defined using statistical relations between neural activity recordings allowing for directed and undirected edges \citep{friston}. Finally, effective connectivity is the causal relation between brain regions \citep{fornito2013graph}. Analysis of the network is achieved by investigating either the topological or dynamical properties \citep{revbull}. Robust and well integrated networks appear to share several similar characteristics e.g. small-worldness, power law degree distribution, modulate hierarchal structure \citep{albert2002statistical}. While it is not currently certain if the brain replicates all these properties, researchers attempt to quantify illnesses through observing changes in certain network properties, when compared to healthy comparisons \citep{revbull}. \\
\\
Studies into brain diseases such as Alzheimer, Parkinson and Multiple Sclerosis have found progress in the application of network theory. Researchers investigate and quantify the effect the degenerative diseases have on the brain in terms of graph theoretic measures. MS patients are observed to have a statistically significant reduction in network communicability \citep{MS1}. Shortest path length (SPL), global efficiency (GB) and network transitivity, all topological properties of a network, were observed to be diminished in MS patients when compared to healthy volunteers, suggesting that a break down in network integrity play a role in the cognitive decline of MS sufferers \citep{MS2}. In further support of \citet{MS2} analysis, MS patients were again observed with decreased network strength, transitivity and GE, with an increase in the SPL network wide \citep{MS3}. Interestingly, the general structural network architecture was found to be preserved, however a decrease in the strength of connections was apparent when compared with healthy controls \citep{MS3}. A study on Alzheimer's disease (AD) was completed by \citet{alz}, in which they investigated the effect of the disease on global network and sub-network scales. At the global network scale, no differences where revealed between AD and healthy controls. However, on the sub-network scale, AD patients appear to have a reduction of the small worldness of the networks with diminished integration. \citet{alz} also investigated other measures they defined as network entropy and complexity, and observed a reduction in the structural diversity and the intricacy of connections patterns. These findings lend support to the notion of network integrity playing an important role in brain health, in that, robust and well integrated networks are seemingly correlated with optimal cognition \citep{revbull}. 

\subsection{Functional Connectivity Network Studies}
Functional connectivity networks are maps of the interactions between brain regions. Building functional connectivity networks is achieved in a range of different manners, with no one standardized procedure yet found \citep{fornito2013graph}. Finding connections between nodes, whether that be sensor based or source based (more on this later), requires the application of a functional connectivity measure. Several measures exist within the literature [Mutual Information, Transfer Entropy, Granger Causality, Pearson Correlation, Partial Correlation, Coherence, Synchronization Likelihood] with each measure applied in a different manner. The pros and cons of functional connectivity measures, and interpretation of found connections have been the focus of much research \citep{jal1,darvas,wangrev}. Complex network analysis performed on functional connectivity networks provides insight into the spatial and temporal dynamics of the brain. Similar to structural networks and their application to various diseases of the brain, researchers have developed functional connectivity networks, constructed using recordings of neural activation (MEG, EEG, fMRI), to explore any changes in brain dynamics that may be present as a result of illness \citep{stamfc, lynfc}.\\
\\ 
Functional connectivity networks have also found application in anesthesia research. By recording brain activity using M/EEG, while patients undergo anesthesia, functional connectivity networks are constructed out of the resulting times series. Segmentation of the time series data, into a set of smaller intervals, enables the construction of dynamical functional connectivity that evolve over time. Analysis of the dynamic network can provide insight into how functional connectivity in the brain changes as a transition from a conscious to unconscious state occurs. \citet{leeold} explored the directionality of frontoparietal networks of patients place under general anesthesia using propofol. A total of 10 subjects had their brain activity recorded using EEG whilst transitioning from conscious to unconscious. The cross dependence was used as a measure of functional connectivity to estimate the feedforward and feedback directionality of frontoparietal network. Interestingly, at loss of consciousness (LOC) a sharp decline in the feedback directionality was observed, dropping to the same magnitude of the feedforward, which remained constant except at the LOC and return of consciousness (ROC). At the ROC, a spike in the feedback and feedforward directionality occurs, with the feedback slowly increasing with time. The results are suggestive that a loss of consciousness is correlated with changes in the functional connectivity of the frontoparietal lobe. The feedback of information is seemingly important in the maintenance of conscious awareness \citep{leeold}.\\
\\
In a separate EEG study on the effects of nitrous oxide as a anesthetic by \citet{kuhl}, they observed a similar reduction in the functional connectivity in the parietal area. They measured the EEG of 20 subjects who had increasing levels of N$_2$O administered until a loss of responsiveness occurred. Their sensor based EEG functional connectivity networks were constructed using GE and GC as a connectivity measure. In addition to the reductions in functional connectivity observed, \citet{kuhl} found that  N$_2$O seemingly perturbs connectivity over a range of spatial scales, concluding that the spatial dependent effects could be the key to anesthetic action. It is clear that building functional connectivity networks provide a unique window through which to view the brain in illness, health and varied conscious states. However, we must be cautious in our interpretations of observed a-typical network topologies and assigning clinical relevance. Many of the modalities employed in constructing brain networks have not been extensively tested at each step in the process, thus any clinical statements should be view within the context of a young and growing field \citep{fornito2013graph, friston}.


\subsection{Brain Nodes}
Creating brain networks requires the choice of a node (vertices) and edge definition. The possible methods in defining nodes in a functional connectivity studies is diverse, but can generally be split into two main categories, voxel based and sensor based. Voxel based can be described as defining nodes in measurement space, such as in FMRI imaging. Sensor network analysis is performed on the neural activation patterns recorded at each sensor location on the scalp \citep{fornito2013graph}. Sensor based nodes are generally used in M/EEG studies and can either be localized to reconstructed sources or assigned directly to the sensors themselves \citep{fall}. Sensor based nodes unfortunately suffer from volume conduction, which is where the measured potential at the sensor is some combination of multiple current sources within the brain \citep{eeg97}. Volume conduction describes the issue of measuring electromagnetic signals through mediums of differing electromagnetic properties. For a signal to propagate through the brain to a sensor it will pass through several different biological materials (brain matter, cerebral spinal fluid, skin and bone), altering the strength of the received signal. Thus the measured potential, is not a true representation of the underlying activity \citep{sens2}. The general approach to dealing with volume conduction effects is source localization, with many different algorithms existing in the literature. The complexity of source localization technique ranges significantly, with the quality of some spatial filtering models heavily dependent on how detailed prior anatomical information is \citep{gross}. If we wish to avoid applying complicated solutions volume conduction can be ignored. The consequences of ignoring the issue will depend ultimately on the research questions attempting to be answered \citep{fall}. If the aim is to investigate the functional interactions between specific regions of the brain, then it is sensible to attempt to solve the volume conduction issue through source localization. If the aim like that of this research thesis, is to learn the network structure present in M/EEG data, without any prior assumptions, then volume conduction can be placed aside for the time being. [REVISE]
 
\subsection{Source Localization}
Source based network analysis attempts to solve the inverse problem of source localization and infer the physical origin of the recorded signal in the brain \citep{revbull}. The functional networks of interest in this research thesis are based upon EEG and MEG recordings. EEG and MEG analysis is generally performed in source space, where the recordings from the scalp (M/EEG) are localized to a source within the brain. The primary motivation for this, is to better determine what regions of the brain are connected functionally during a task and the possibility of clinically significant information. Source localization however, is inherently fraught with difficulties due to its ill-posed nature. Locating the origin of electromagnetic fields within the brain volume is an inverse problem, where the 2-D potential map, in the case of EEG, is localized within the cranium. Currently, there exists a myriad of source localization algorithms, however the accuracy of the functional interaction results obtained has not been tested extensively \citep{yao}.  \citet{bradsl} evaluated several EEG source localization methods with multiple cortical sources, to investigate whether the algorithms could reconstruct known input sources. The algorithms performed well when only a single source was present. However, even when as few as four sources were present (maximum number tested), the ability to resolve the sources accurately, suffered. In another study completed by \citet{yao}, where the objective was to compare several localization methods using simulated EEG data, they found a similar diversity in the success of the each model, with one in particular being the most successful. Due to the ill-posed nature of the source localization, a range of priori assumptions must be made to reduce the number of non-realistic solutions down to something more manageable, the data alone is not sufficient \citep{wagner}. Despite the drawbacks, the application of source localization has found much success in network analysis of the brain. However, caution is still warranted in blindly applying these tools.\\
\\
With the research aim of the project to construct a robust network inference tool using partial correlations among time series data, it is hoped that spatial filtering and source localization of MEG/EEG recordings will prove to be unnecessary, serving only to add additional layers of complexity and increasing the risk of bias. Starting at square one with a purely sensor-based analysis is therefore the best and more reasonable starting point as it is model- and assumption-free, helping to reduce the difficulty of the task ahead. It is noted that there is evidence that volume conduction within the brain renders non-filtered or non-localized sensor-based analysis moot due to the way it acts to alter and combine signals \citep{sens2}. This may lead to spurious correlations upon analysis. \citet{sens1} suggests volume conductance and reference dependence may make it impossible to draw reasonable conclusions from a purely sensor-based analysis.\\
\\
We have shown that several of the oft employed localization methods, despite being valuable, are prone to error and misinterpretation. It is our belief then, in line with the conclusions expressed in \citep{stnode} that if we are fundamentally interested in inferring any networks present within the data, that we enter into the study with no a priori assumptions. Whilst \citet{stnode} argued for voxel-based networks within the context of fMRI research, their argument fits appropriately within our study ``\emph{because the voxel-wise approach does not require implementing any a priori assumptions regarding what constitutes the “right” node, the approach is fundamentally unbiased. Therefore, the approach allows the data to speak for itself}''. When there exists no current right answer to network inference problem, then the reasonable approach is the one which introduces as little bias as possible. The best way to achieve this with M/EEG data is to explore the time series at the sensor level without any prior constraints. Again, inferring the network will be best achieved using pairwise partial correlations between nodes.

\subsection{Partial Correlation}
The underlying aim of this research thesis is to infer network structure within the data, in such a way as to minimize the possibility of introducing bias. We wish to let the data speak for itself and not be guided and/or corrupted by any prior assumptions. In keeping with this goal, the choice of partial correlation as a tool to infer networks follows naturally. Partial correlation is a data driven connectivity measure, requiring only a few prior assumptions about the nature of the data, be that it is continuous, approximately gaussian and relations between variables linear \citep{linn}. Partial correlation is used to infer the direct connections between two nodes in a network. It calculates the relationship between the two nodes with all other nodes in the network regressed out, thus avoiding possibly spurious correlations \citep{zalesky2012use}. In this way, it is superior to standard correlation estimates using the Pearson correlation coefficient, which will include non-direct edges. Figure \ref{b2} is a graphical example of a small network. If nodes X and Y are strongly correlated, and nodes Y and Z are also strongly correlated, then the standard correlation coefficient would assume X and Z are also correlated. However, in reality that may not be the case, and the situation can occur in brain networks. One node may be well correlated with another through a common neighbor, but they may not actually be connected outside of the shared neighbor connection. Partial correlation accounts for this by controlling for the influence of the neighbor when investigating the relation between X and Z. Partial correlation has found to be a useful tool in network inference within fMRI and EEG \citep{marrelec2006partial, wangpc}.\\
\\
\begin{figure}%[h!]%[H]
\centering\includegraphics[scale = 0.8]{brain2.png}%width=\linewidth
\caption{Correlation between nodes. Red is inferred connection between X and Z caused by the common neighbor, Y.}
\label{b2}
\end{figure}\\ 
Many methods exists within the literature to learn networks within data. What makes partial correlation a more sensible choice for this research, is the ability to generate direct connections between nodes without relying on a range on prior assumptions. Mutual information is another interesting connectivity measure, based in information theory \citep{mackay}. However, it suffers from the need to have either a known underlying probability distribution for the data, or one that is generated using the data in the form of a frequency distribution. The heuristic manner in which the data is binned can influence the MI estimate obtained \citep{MIC1}. Partial correlation doesn't suffer from this issue, making it a better choice for our purposes.
\\
\\
As we touched on earlier in this review, a core step performed in many functional connectivity analysis is thresholding the connectivity matrix. Upon completing the connectivity measure analysis, a symmetric (undirected) or non-symmetric(directed) matrix will contain the values of the pairwise connections between nodes. Provided with the connectivity matrix the process of thresholding occurs, which enables the construction of a binary adjacency matrix which graph metrics can be applied. Generally, a non-zero connectivity will exist for all pairs of nodes, so the goal is to effectively prune the graph so that only meaningful connections remain. Unfortunately, no justifiable method yet exists to guide the thresholding value, which typically results in multiple iterations of the experiment performed over a range of cut off values, which is computationally inefficient and a waste of resources. In our research, we are using partial correlations as a connectivity measure. Interestingly, the very nature of partial correlation acts to remove spurious edges that may be present due to neighbor to neighbor connections by discounting conditional dependencies. Given that, it is possible that thresholding to achieve a sparse network is not required. Furthermore, if we do not wish to construct a binary adjacency matrix on which to apply graph metrics, and a symmetric weighted network is suitable for our analysis, then it seems a sensible choice to not remove possibly valid edges by applying an ill-founded threshold. Fortunately, many graph metrics that are of interest in this thesis `clustering, degree distribution, eigenvector centrality' have all been shown to be generalizable from the standard binary adjacency matrix form, to a weighted adjacency network \citep{fornito2013graph, wn2, wnsal}. Enabling a graph metric analysis of any obtained partial correlation networks. 

\pagebreak
%\citep{zalesky2012use}

%\nocite{*}
%\bibliographystyle{apalike}
\bibliographystyle{plainnat}
\bibliography{test}




\end{document}


%%%%%%%%%%%%%%%%%%%%%%%%%%%%%%%%%%%%%%%%%%%%%%%%%%%%%%%%%%%%%%%%%%%%%%%%%%%%%%%%%%%%%%%%%%%%%%%%%%%%%%%%%%%%%%%%%%%%5
%%%%%%%%%%%%%%%%%%%%%%%%%%%%%%%%%%%%%%%%%%%%%%%%%%%%%%%%%%%%%%%%%%%%%%%%%%%%%%%%%%%%%%%%%%%%%%%%%%%%%%%%%%%%%%%%%%%%5
%%%%%%%%%%%%%%%%%%%%%%%%%%%%%%%%%%%%%%%%%%%%%%%%%%%%%%%%%%%%%%%%%%%%%%%%%%%%%%%%%%%%%%%%%%%%%%%%%%%%%%%%%%%%%%%%%%%%5

\begin{figure}[h!]%[H]
\centering\includegraphics[scale = 0.8]{PLANCK.png}%width=\linewidth
\caption{Planck CMB power spectrum (from Planck 2015)}
\label{POW}
\end{figure} 


Investigations into the dynamical changes in functional connectivity in the human brain induced by anesthesia, is an active and diverse area of research. The studies rely on the ability to measure brain activity using EEG/MEG and FMRI. From these activity recordings, researchers attempt to develop precise and accurate representations of both the location of the neural populations and the communication occurring between them. The desire is to be able to make quantitative statements about the changes in consciousness and the pathway through which these changes come about. Given the complexity of consciousness and the brain in general it is no surprise that a range of models have been developed to attempt to achieve these goals, with no single solution yet existing.
\\
\\
There exists at least four main problems that are of concern to researchers hoping to derive clinical information using functional connectivity studies and they are; the forward model/solution, inverse model/solution, network learning and quantitative tools. The discussion of the mentioned problems provides a good start point for the review of the relevant literature, given that, a detailed description of each follows. The forward model is an attempt at modeling the both physical topology of the head e.g. shape and volume and the electromagnetic properties of the various tissues e.g. skin, bone, muscle and brain matter. The second problem is the inverse solution, which itself is intimately tied to the first issue.  The inverse problem can be summarized as follows; given a potential map across the surface of the scalp, obtained through EEG recordings, how can the origin of the signals be located in source space? In other words, how does one go from a 2-dimensional surface map to a 3-dimensional source localization? The solution to the inverse problem is complex and with no one size fits all, with each solution requiring implementation of various constraints to reduce the set of possible solutions from infinity to a physically realizable one. 
\\
\\
Network learning is the third problem and covers the issue of building the brain network up from a combination of nodes and edges. The manner in which the nodes and edges are defined can ultimately govern the experimental results obtained. Constructing a brain network can be done in many different ways leading to differing results [REFERENCE]. Functional connectivity can be thought of as edges that have some temporal behavior associated with them, which introduces an additional layer of complexity. The connections are no longer static and can have directional dependencies [REFERENCE], therefore the choice of time scale and the way the connectivity is defined matters significantly [REFERENCE]. The network learning problem is ultimately summarized as the issues that can be found in choosing the appropriate node/edge model for the network which takes into account or relevant spatio-temporal information.
\\
\\
The final problem is that of the quantitative tools used to derive information from the dynamical and topological behavior. Investigating network properties employs the use of a large set of tools ranging from statistical mechanics, graph theory to machine learning A survey of all the different techniques is beyond the scope of this thesis, however an exhaustive review can be found in [REFERENCE], [REFERENCE] and [REFERENCE. Ultimately, the reviews converge on the core issue, that being the results of any particular study will reflect the initial choice of quantitative assessment.


comes when attempting to locate the sources of the recorded signals. MEG and EEG take time series recordings at the sensor level, from the recordings researchers attempt to infer the location within the brain at which the source originated. The solution is ill-defined and inverse in nature, with an infinite number or possible solutions present that are needed to be constrained 
\\
\\
Using an appropriate functional connectivity measures is not the only issue faced when modeling the human brain as a complex network. The way in which the network itself is defined is an area of concern as \citet{fornito2013graph} found when a review was completed on some of the current ways in which brain networks are constructed. Once a particular model is adopted, the functional connectivity between brain regions can be observed and quantified within that framework. 



%%%
% SOUNDS LIKE YOU'VE GOT A LOT OF LITERATURE TO REVIEW, SON!
% Curious about whether consciousness is actually an emergent phenomena arising due to complex interactions between large neural networks.
% Investigating the changes as consciousness drifts when drugged up sounds like a good way to start!
%%% 

\subsection{Stages of Research}
The current plan for this research thesis is separated into three stages, with each building upon the previous. I felt it is a good idea to flesh out the plan in detail here for my own record and to keep track of progress.
\subsubsection{Inverse Problem - Source Localization and Beamforming}
The first and mosts fundamental step is to investigate the inverse problem of source localization and develop a good understanding of the issue and relating literature. Once this has been done, I am to develop and or build upon currently available source localization methods tailored to the MEG/EEG issue. At its core, the problem can be summarized as follows; given electric potential readings in the case of EEG or Magnetic fields in MEG at the level of the scalp, the aim is to determine the origin of the signal within the brain. Unfortunately, as we are attempting to work in reverse, from the measured field values, to the source origin, there could exist infinitely many combinations of sources within the volume that could give rise to the measured field gradient. To determine the source, assumptions must be made about the sources that enable a vast minimization of the number of source combinations that could provide the observed data.. In doing this, the method of minimizing will inevitably influence the final result. Thus, it is essential that the assumptions and constraints made are physically sensible and promote good results. How this is done will be the focus of this stage. Preliminary readings do not inspire confidence, as it appears that there exists many different methods and no consensus upon which works best. Much ongoing research is being done on the ill-posed inverse problem.

\subsubsection{LCMV Beamforming}
The objective of beamforming within the brain is to best estimate the location of a current dipole source at some location given measurements at multiple locations of either or in combination, the magnetic field outside the head or the electric potential along the scalp surface.

\subsubsection{Learning the network}



The aim of this research thesis is to investigate the dynamical changes in functional brain networks in humans as subjects are placed under and brought out of chemically induced anesthesia. By inducing unconscious states using chemical anesthesia, Nitrogen Oxide $N_2O$ and Xenon \emph{Xe}, the subject's brain activity is recorded using EEG and MEG simultaneously to provide a suite of data. The activity data is then reduced and converted into functional brain network signals to be analysed using mathematical and physical investigative means. The desire is to observe any dynamical changes in the functional connectivity of the brain networks as the subject experiences a loss of consciousness and subsequent return of a conscious cognitive state. This research aims to begin to quantify \emph{consciousness} as an emergent property of dynamical brain networks.
\\
\\
We begin the research thesis with a review of the relevant literature, detailing the core ideas and concepts required to understand the proceeding sections. Then, moving onto the methods where we provide a brief explanation of the data collection, and a detailed description of the physics and statistical tools used to investigate the network dynamics. The results obtained are then presented and subsequently discussed, followed by a conclusion where we address future research avenues.

\section{Introduction \& Literature Review}
The nature of consciousness is one of the greatest unsolved mysteries in modern science. Our attempt to quantify and derive some measure for our conscious experience has led to the development of many sophisticated technologies and methodologies. 


Several core ideas need to be addressed before the research component of this thesis is explored. The core ideas are: defining nodes and edges in brain networks, data collection MEG/ECG sampling, statistical mechanics of complex networks and consciousness as a complex emergent phenomenon.\\
\\
Investigations into the nature of consciousness in the brain has lead to a myriad of tools and techniques developed to attempt to best quantify levels of awareness and begin to piece together a description of conscious states. One such method is the idea of casting the brain as a complex network. The brain is described as a network made up of various nodes and edges. The way in which the nodes and edges in the brain network are defined ultimately characterize what will be studied. The brain network can be defined at differing resolutions including; individual collections of neurons which are connected via their synaptic interactions, larger groupings of neurons defined by a volume of brain matter or on the larger scale, individual regions of the brain connected through some physical or functional connectivity (Parietal Lobe, Frontal Lobe, Hippocampus ect.). Modeling the brain as a network allows the application of graph theory and statistical mechanics, enabling  a quantification of interesting topological and statistical features.\\
\\
One on-going goal in neuroscience is to investigate atypical behavior and attempt to find some underlying structure or functional properties that lead to the observed deviation from the norm. The ability to characterize varied conscious states as a property of the underlying network topology has immense clinical value. Much research has been done in patients with psychological and functional disorders such as schizophrenia and epilepsy. The studies explore the functional connectivity of a range of patients and attempt to quantify  some measure of the complex network that deviates from that found in healthy patients [EXPAND \& REF].\\
\\
Interestingly, whilst network studies of the brain are proving immensely valuable, there exists a range of issues that need to be better understood before some standard becomes available. Several key issues are the subject of active research and will be explored here in more detail, they are: Source Localization, Network Learning (Nodes, Edges) and Functional connectivity measures. 


The method employed within this research thesis, is built upon previous works completed by \dots. The functional connectivity of \dots The nodes in the brain network will be defined by segmented brain matter volumes defined using beam forming methods and MEG/EEG data. The interaction between the nodes will be characterized by the correlation in the time series data obtained during the MEG and EEG experiments.

\subsubsection*{Node and Edge}
To define the network nodes, source localization will be used to best approximate the source of the signal generation in the brain. The EEG and MEG data will then be segmented into numerous 'instance of time' with the appropriate considerations applied. The functional edges between the nodes are then generated using the partial correlation measures. This will give insight into how one region of the brain is changing with the other. A suite of statistical tools will then be applied to quantify the network dynamics and topology. An example of one graphical measure is the eigenvector centrality, which effectively determines the importance of a node in a network. Interestingly, many of the measures that are used to quantify binary adjacency matrices can still be applied to a weighted network. For example, the degree is no longer some  integer value, but a some of the edge strengths, in the example case, it could be the partial correlation values of all the nodes connected to the region of interest. [REF]

\subsection{Consciousness emergence from complexity - for own reading purposes}

\subsection{Complexity - for own reading purposes}
A system that is maximally complex is in a state that has the largest potential for information processing and complex emergent behavior. When a system is maximally complex, it requires the greatest amount of information to define and understand. This provides the ability for the system to exhibit complex behaviors that result in higher level information processing (Kolmogorov Complexity somewhat related to this idea.)

\section{Methods}

\subsection{The source localization problem}
Source localization can be summarized as an attempt to determine the source of a signal from some resulting pattern of observations. In the context of MEG/EEG brain data, source localization is the problem of determining the source of the signal within the brain volume, from the surface measurements on the scalp surface. The source localization is an inverse problem where we attempt to work back from an observed data distribution to determine the physical source. Clinically, the ability to localize the brain activity to some desired region and spatial resolution cannot be understated. However, within the context of this current research thesis it is not as clear  cut. The desire is to characterize the properties of dynamical functional brain networks and attempt to observe some topological and statistical deviations in the networks of differing conscious states.\\ 
\\
The ability to resolve the source of the data, is undoubtedly important as it will provide a higher level of diagnostic information. Despite this, in effort to justify the time and computational resources devoted to source localization, it is currently unclear as to whether broaching the inverse problem is a worthwhile pursuit for this project. The ability to investigate dynamic functional networks is completely possible at the sensor level, and is not inhibited by a lack of source. The main goal of the research as it is currently constructed is to be able to derive quantifiable topological changes in the network structure as it changes in time. Removing the brain space source localization will  only remove the ability to make quantitative statements about which portions of the brain are communicating and synchronizing their activity, but the network changes do not depend on the ability to locate the source. Perhaps a better approach is to investigate some preliminary studies  into the changing network topologies at the sensor level, then time permitting, build the source localization in. The remaining issue is which source localization method to use as there exists no ''one size fits all'' solution, with the currently proposed methods each having a suite of difficulties and  benefits. Much more reading is needed. 

\subsection{Method - Data Collection}
The data used within this research thesis was obtained by simultaneous collection of MEG and EEG signals from patients. The subjects were placed under chemically induced anesthesia using Xe and $N_2 O$ gas. The simultaneous collection of the EEG and MEG enables are more accurate localization of the brain region responsible for the signal. To investigate the functional connectivity of brain networks, nodes and edges must be defined. The nodes of the functional networks are defined as voxels of brain matter which are located using beamforming techniques (MUST EXPLAIN BEAMFORMING AND THE VOXEL SIZE). The functional connections between nodes, edges, are defined (CURRENTLY) using the inverse covariance matrix of the data sets. By using the inverse covariance, we can observe how the individual nodes are correlated with one another, whilst accounting for the correlations between the other nodes, which can dramatically change the functional  connectivity strength (REFERENCE AND DETAILED EXPLANATION OF CORRELATION AND INVERSE COVARIANCE MATRICES). Once the nodes have been defined using beamforming, the edges can be determined. However this is not a trivial process, as the way in which the data is parsed  and analyzed will ultimately dictate the results obtained, more details in the following section.

\subsection{Method - Data Analysis - Psuedocode explanation}


\begin{itemize}
\item Define nodes as volumes of brain matter by using beam forming techniques and  the position of the MEG/EEG sensors.
\item Using the time series data, segment it using statistical techniques, enabling the  temporal and spatial analysis of brain activity.
\item Apply other graphical/topological measures to the functional network to derive  important network characteristics.
\item If time permits extend research to incorporate machine learning - develop a tool  that can analyse a set of time series and state whether the subject is in conscious or unconscious state.
\item If time permits extend the analysis to incorporate an investigation of network  entropy (if possible with current understanding)
\end{itemize}
The second point in the above list is crucial. How the time series of the MEG and EEG are  parsed into segments is of particular importance. The time series data (n-number of magnetic and electric field measures) needs to be segmented so that the correlation/inverse-covariance can be determined at different points in time. But it must be segmented in a way which maximizes the usefullness and quality of the obtained results. If no segmenting is done, then we essentially arrive at nxn inverse-covariance matrix for the  whole time, which will not provide meaningful dynamical information. If we segment each data stream into n-number of points, where n is equal to the total array length, then it is not possible to define correlation with just a single data point per node, which can be interpreted as a viewing  the brain for a brief moment in time and attempting to infer how each component is correlated with the other. Which is to say, that it is nonsensical. Now, there must exist and optimal parse length between the two extremes which preserves best the dynamic behavior of the functional  connectivity. As the time series ultimately form oscillating waves, it is possible that the answered be guided by wave dynamics. Jack White of Swinburne mentioned the coherence length of each wave set perhaps be the parse guide. The answer to this question is still being investigated.
\section{Results}
Project 4 Description - Author David Liley - Neural inertia in the breakdown and recovery of functional brain networks
\\
\\

We will test the hypothesis that the excitatory anesthetic agents nitrous oxide and xenon  will exhibit a concentration dependent hysteresis in organizational measures of specific brain networks that, at rest, are well defined by particular functional topological features. The outcomes of this set of experiments will allow us to determine i) to what extent neural inertia can be understood  in terms of the functional breakdown and recovery of large scale brain neural networks, and ii) the degree to which the barriers to brain state transitions are independent of the underlying neurochemical perturbations.

The excitatory anesthetic agents, exemplified by nitrous oxide, xenon and ketamine, represent  an interesting class of drug. Unlike the clinically more ubiquitous drugs, such as propofol and the halogenated ethers (e.g. sevoflurane), which act principally by enhancing inhibitory activity in the CNS, excitatory, or dissociative agents, act primarily by reducing excitatory neural activity. While operationally both classes of agent can lead to equivalent clinical endpoints they do so by distinct molecular level targets of action. The phenomenon of  neural inertia will be considered a central organizing feature in cognition if we show that it is independent of the specific underlying cause of any perturbation to consciousness. On this basis we will measure changes in functional brain network organization before, during and after the inhalation  of xenon and nitrous oxide in thirty human volunteers.

We will use a novel and definitive approach to quantifying changes in functional brain network  architecture during excitatory anesthesia. This will involve the administration of nitrous oxide and xenon to healthy participants in a balanced two-way crossover design during which they will have simultaneous MEG and EEG recorded. Xenon and nitrous oxide are specifically chosen because i)  together with end-tidal gas measurement and pharmacokinetic modelling, their brain effect site concentrations can be continuously measured and ii) pilot experiments have indicated asymmetric changes in total, and band limited, EEG power during concentration increases and decreases. State-of-the-art inverse modelling methods will be applied to the simultaneously recorded MEG and EEG in order to estimate source-level cortical and subcortical  neuronal activity. This high-resolution source-level analysis will be used to define nodes (vertices) and connections (edges) for subsequent graph theoretical estimates of concentration dependent measures of functional and effective connectivity. 

The experiments and the subsequent data driven computational analyses will provide an  unprecedented high resolution panorama of the changes in brain activity that occur in response to systematic neurochemical manipulations of the state of consciousness.
\section{Discussion}

\section{Project Outline}
The aim of this research thesis is to investigate the dynamical behavior of functional brain networks of patients in varying states of consciousness. The functional brain networks will be defined by applying methods from statistical mechanics and graph theory to MEG and EEG data. The data was collected simultaneously from patients who were placed under chemically induced anesthesia, using Xenon and Nitrous Oxide gas. The desire is to be able to characterize the effects of anesthetics and explore the pathways by which they operate to reduce consciousness. The patients are monitored prior, during and post anesthesia, providing a temporally and spatially resolved signal of neural  activity in varies regions of the brain. By defining and modeling the activity in the form of functional brain networks, the goal is to be able to quantify states of differing consciousness by the topological or statistical properties of the dynamic networks. The research will be completed under the supervision of Prof. Damien Hicks and Prof. of Swinburne University of Technology.

\section{Outline of Literature Review}
I need to sketch out the general brain networks, how and why they are used, both structural and functional. Then discuss functional connectivity in general, this leads to functional connectivity studies in consciousness. Then to sensor and source based, why sensor based should be used? Then focus on partial correlations as connectivity measures.
\subsection{General Themes}
\begin{itemize}
\item What is a brain network?
\item How are they built?
\item Why are they used?
\end{itemize}
\subsection{Functional Connectivity}
\begin{itemize}
\item What is functional connectivity?
\item What studies is it used in?
\item Example of measures
\end{itemize}
\subsection{Anesthesia studies - Sensor/Source Based - Partial correlation}
\begin{itemize}
\item Studies involving functional connectivity
\item Sensor or source based?
\item Partial correlation over other measures
\end{itemize}
\section{Sensor versus Source Based}
The main motivation for this research thesis is to investigate whether sensor based (time-series) MEG/EEG analysis, can provide statistically meaningful information on functional connectivity changes in neural activity recordings. EEG and MEG analysis is generally performed in source space, where the recordings from the scalp (EEG) or above (MEG) are localized to a source within the brain. The primary motivation for this, is to better determine what regions of the brain are connected functionally during a task. Source localization however, is inherently fraught with difficulties due to its ill-posed nature. Locating the origin of electromagnetic fields within the brain volume is an inverse problem, where the 2-D potential map, in the case of EEG, is localized within the cranium. Currently, there exists a myriad of source localization algorithms, however the accuracy of the activity results obtained has not been tested extensively [REF].  REF ET AL evaluated several EEG source localization algorithms with multiple cortical sources, to investigate whether the  However, due to the ill-posed nature of the source localization (inverse problem), a range of priori assumptions must be made to reduce the infinite number of solutions down to something more manageable.\\
\\ 
Pros to using source localization: Provides information on underlying physiology - possible clinical information.\\
\\
Cons to using source localization: broad range of procedures - little consensus on the best - no one size fits all - results dependent somewhat on model used an priori assumptions - computational complex - difficulty in interpreting results.  
\\
Pros to using sensor based: Simple and more easily implemented - Few, if any assumptions made - Computationally faster - Dealing with raw data so no introduction of bias from model assumptions - seemingly a good first approximation to form build further localization studies on
\\
\\
Cons to using sensor based: Little, if any physiological information - no clinical information - possibly not informative at all

It is of the belief of the author, that these higher level and complex localization procedures would be complimented well by low level sensor based analysis. Such a base level analysis
The ability to model the brain as a complex network enables the quantification of structural and temporal properties that had previously been unfeasible (Sporns 2010). Despite the brains receptivity to being modeled in this way, it is not without experimental drawbacks. In a paper by Fornito and Zalesky (2013), they access the potential of complex networks as a tool for brain research. They discuss several limitations that are currently faced at various stages of modeling that can influence results. The first and major issue that is meets any team is the ability to define the network nodes. The brain, as mentioned prior, is made up of billions of neurons and trillions of connections, now to model the brain at the neuron scale would be computationally unfeasible.
