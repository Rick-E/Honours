\documentclass{article}
%\twocolumn


\usepackage{graphicx}
\usepackage[justification=centering]{caption}
\graphicspath{Figss/}
\usepackage{amsmath,amssymb}
\usepackage{a4wide}
\def\ni{\noindent}
\begin{document}

\title{\bf{Dynamics of Functional Brain Networks}}
\author{Rick Evertz 9548866\\
Supervisor: Prof. Damien Hicks\\
Swinburne University of Technology}
\date{\today}
\maketitle
\abstract 
\ni Investigating brain networks of people in conscious and unconscious states.
\pagebreak

\tableofcontents
\pagebreak
\section{Introduction}
\subsection{Stages of Research}
The current plan for this research thesis is separated into three stages, with each building upon the previous. I felt it is a good idea to flesh out the plan in detail here for my own record and to keep track of progress.
\subsubsection{Inverse Problem - Source Localization and Beamforming}
The first and mosts fundamental step is to investigate the inverse problem of source localization and develop a good understanding of the issue and relating literature. Once this has been done, I am to develop and or build upon currently available source localization methods tailored to the MEG/EEG issue. At its core, the problem can be summarized as follows; given electric potential readings in the case of EEG or Magnetic fields in MEG at the level of the scalp, the aim is to determine the origin of the signal within the brain. Unfortunately, as we are attempting to work in reverse, from the measured field values, to the source origin, there could exist infinitely many combinations of sources within the volume that could give rise to the measured field gradient. To determine the source, assumptions must be made about the sources that enable a vast minimization of the number of source combinations that could provide the observed data. In doing this, the method of minimizing will inevitably influence the final result. Thus, it is essential that the assumptions and constraints made are physically sensible and promote good results. How this is done will be the focus of this stage. Preliminary readings do not inspire confidence, as it appears that there exists many different methods and no consensus upon which works best. Much ongoing research is being done on the ill-posed inverse problem.

\subsubsection{Learning the network}

 


The aim of this research thesis is to investigate the dynamical changes in functional brain networks in humans as subjects are placed under and brought out of chemically induced anesthesia. By inducing unconscious states using chemical anesthesia, Nitrogen Oxide $N_2O$ and Xenon \emph{Xe}, the subject's brain activity is recorded using EEG and MEG simultaneously to provide a suite of data. The activity data is then reduced and converted into functional brain network signals to be analysed using mathematical and physical investigative means. The desire is to observe any dynamical changes in the functional connectivity of the brain networks as the subject experiences a loss of consciousness and subsequent return of a conscious cognitive state. This research aims to begin to quantify \emph{consciousness} as an emergent property of dynamical brain networks.
\\
\\
We begin the research thesis with a review of the relevant literature, detailing the core ideas and concepts required to understand the proceeding sections. Then, moving onto the methods where we provide a brief explanation of the data collection, and a detailed description of the physics and statistical tools used to investigate the network dynamics. The results obtained are then presented and subsequently discussed, followed by a conclusion where we address future research avenues.

\section{Literature Review}
Several core ideas need to be addressed before the research component of this thesis is explored. The core ideas are: defining nodes and edges in brain networks, data collection MEG/ECG sampling, statistical mechanics of complex networks and consciousness as a complex emergent phenomenon.\\
\\
Investigations into the emergence of consciousness in the brain has lead to a myriad of techniques used in attempt to quantify levels of awareness and begin to piece together a picture of conscious states. One such tool is the idea of casting the brain and its various components as a complex network. The brain is described as a network made up of various nodes and edges. The way in which the nodes and edges in the brain network are defined ultimately characterize what will be studied. The brain network can be defined at differing resolutions including; individual collections of neurons which are connected via their synaptic interactions, larger groupings of neurons defined by a volume of brain matter or on the larger scale, individual regions of the brain itself (Parietal Lobe, Frontal Lobe, Hippocampus ect.). Modeling the brain as a network allows the application of graph theory and statistical mechanics, enabling a quantification of various network properties and behaviors.\\
\\
The method employed within this research thesis, is built upon previous works completed by \dots. The functional connectivity of \dots The nodes in the brain network will be defined by segmented brain matter volumes defined using beam forming methods and MEG/EEG data. The interaction between the nodes will be characterized by the correlation in the time series data obtained during the MEG and EEG experiments.

\subsubsection*{Node and Edge}


\subsection{Consciousness emergence from complexity - for own reading purposes}

\subsection{Complexity - for own reading purposes}
A system that is maximally complex is in a state that has the largest potential for information processing and complex emergent behavior. When a system is maximally complex, it requires the greatest amount of information to define and understand. This provides the ability for the system to exhibit complex behaviors that result in higher level information processing (Kolmogorov Complexity somewhat related to this idea.)

\section{Methods}

\subsection{The source localization problem}
Source localization can be summarized as an attempt to determine the source of a signal from some resulting pattern of observations. In the context of MEG/EEG brain data, source localization is the problem of determining the source of the signal within the brain volume, from the surface measurements on the scalp surface. The source localization is an inverse problem where we attempt to work back from an observed data distribution to determine the physical source. Clinically, the ability to localize the brain activity to some desired region and spatial resolution cannot be understated. However, within the context of this current research thesis it is not as clear cut. The desire is to characterize the properties of dynamical functional brain networks and attempt to observe some topological and statistical deviations in the networks of differing conscious states.\\ 
\\
The ability to resolve the source of the data, is undoubtedly important as it will provide a higher level of diagnostic information. Despite this, in effort to justify the time and computational resources devoted to source localization, it is currently unclear as to whether broaching the inverse problem is a worthwhile pursuit for this project. The ability to investigate dynamic functional networks is completely possible at the sensor level, and is not inhibited by a lack of source. The main goal of the research as it is currently constructed is to be able to derive quantifiable topological changes in the network structure as it changes in time. Removing the brain space source localization will only remove the ability to make quantitative statements about which portions of the brain are communicating and synchronizing their activity, but the network changes do not depend on the ability to locate the source. Perhaps a better approach is to investigate some preliminary studies into the changing network topologies at the sensor level, then time permitting, build the source localization in. The remaining issue is which source localization method to use as there exists no ''one size fits all'' solution, with the currently proposed methods each having a suite of difficulties and benefits. Much more reading is needed. 

\subsection{Method - Data Collection}
The data used within this research thesis was obtained  by simultaneous collection of MEG and EEG signals from patients. The subjects were placed under chemically induced anesthesia using Xe and $N_2 O$ gas. The simultaneous collection of the EEG and MEG enables are more accurate localization of the brain region responsible for the signal. To investigate the functional connectivity of brain networks, nodes and edges must be defined. The nodes of the functional networks are defined as voxels of brain matter which are located using beamforming techniques (MUST EXPLAIN BEAMFORMING AND THE VOXEL SIZE). The functional connections between nodes, edges, are defined (CURRENTLY) using the inverse covariance matrix of the data sets. By using the inverse covariance, we can observe how the individual nodes are correlated with one another, whilst accounting for the correlations between the other nodes, which can dramatically change the functional connectivity strength (REFERENCE AND DETAILED EXPLANATION OF CORRELATION AND INVERSE COVARIANCE MATRICES). Once the nodes have been defined using beamforming, the edges can be determined. However this is not a trivial process, as the way in which the data is parsed and analyzed will ultimately dictate the results obtained, more details in the following section.

\subsection{Method - Data Analysis - Psuedocode explanation}


\begin{itemize}
\item Define nodes as volumes of brain matter by using beam forming techniques and the position of the MEG/EEG sensors.
\item Using the time series data, segment it using statistical techniques, enabling the temporal and spatial analysis of brain activity.
\item Apply other graphical/topological measures to the functional network to derive important network characteristics.
\item If time permits extend research to incorporate machine learning - develop a tool that can analyse a set of time series and state whether the subject is in conscious or unconscious state.
\item If time permits extend the analysis to incorporate an investigation of network entropy (if possible with current understanding)
\end{itemize}
The second point in the above list is crucial. How the time series of the MEG and EEG are parsed into segments is of particular importance. The time series data (n-number of magnetic and electric field measures) needs to be segmented so that the correlation/inverse-covariance can be determined at different points in time. But it must be segmented in a way which maximizes the usefullness and quality of the obtained results. If no segmenting is done, then we essentially arrive at nxn inverse-covariance matrix for the whole time, which will not provide meaningful dynamical information. If we segment each data stream into n-number of points, where n is equal to the total array length, then it is not possible to define correlation with just a single data point per node, which can be interpreted as a viewing the brain for a brief moment in time and attempting to infer how each component is correlated with the other. Which is to say, that it is nonsensical. Now, there must exist and optimal parse length between the two extremes which preserves best the dynamic behavior of the functional connectivity. As the time series ultimately form oscillating waves, it is possible that the answered be guided by wave dynamics. Jack White of Swinburne mentioned the coherence length of each wave set perhaps be the parse guide. The answer to this question is still being investigated.
\section{Results}
Project 4 Description - Author David Liley - Neural inertia in the breakdown and recovery of functional brain networks
\\
\\

We will test the hypothesis that the excitatory anesthetic agents nitrous oxide and xenon will exhibit a concentration dependent hysteresis in organizational measures of specific brain networks that, at rest, are well defined by particular functional topological features.  The outcomes of this set of experiments will allow us to determine i) to what extent neural inertia can be understood in terms of the functional breakdown and recovery of large scale brain neural networks, and ii) the degree to which the barriers to brain state transitions are independent of the underlying neurochemical perturbations.

The excitatory anesthetic agents, exemplified by nitrous oxide, xenon and ketamine, represent an interesting class of drug.  Unlike the clinically more ubiquitous drugs, such as propofol and the halogenated ethers (e.g. sevoflurane), which act principally by enhancing inhibitory activity in the CNS, excitatory, or dissociative agents, act primarily by reducing excitatory neural activity.  While operationally both classes of agent can lead to equivalent clinical endpoints they do so by distinct molecular level targets of action.  The phenomenon of neural inertia will be considered a central organizing feature in cognition if we show that it is independent of the specific underlying cause of any perturbation to consciousness.  On this basis we will measure changes in functional brain network organization before, during and after the inhalation of xenon and nitrous oxide in thirty human volunteers.

We will use a novel and definitive approach to quantifying changes in functional brain network architecture during excitatory anesthesia. This will involve the administration of nitrous oxide and xenon to healthy participants in a balanced two-way crossover design during which they will have simultaneous MEG and EEG recorded.  Xenon and nitrous oxide are specifically chosen because i) together with end-tidal gas measurement and pharmacokinetic modelling, their brain effect site concentrations can be continuously measured and ii) pilot experiments have indicated asymmetric changes in total, and band limited, EEG power during concentration increases and decreases.  State-of-the-art inverse modelling methods will be applied to the simultaneously recorded MEG and EEG in order to estimate source-level cortical and subcortical neuronal activity.  This high-resolution source-level analysis will be used to define nodes (vertices) and connections (edges) for subsequent graph theoretical estimates of concentration dependent measures of functional and effective connectivity.   

The experiments and the subsequent data driven computational analyses will provide an unprecedented high resolution panorama of the changes in brain activity that occur in response to systematic neurochemical manipulations of the state of consciousness.
\section{Discussion}

\section{Project Outline}
The aim of this research thesis is to investigate the dynamical behavior of functional brain networks of patients in varying states of consciousness. The functional brain networks will be defined by applying methods from statistical mechanics and graph theory to MEG and EEG data. The data was collected simultaneously from patients who were placed under chemically induced anesthesia, using Xenon and Nitrous Oxide gas. The desire is to be able to characterize the effects of anesthetics and explore the pathways by which they operate to reduce consciousness. The patients are monitored prior, during and post anesthesia, providing a temporally and spatially resolved signal of neural activity in varies regions of the brain. By defining and modeling the activity in the form of functional brain networks, the goal is to be able to quantify states of differing consciousness by the topological or statistical properties of the dynamic networks. The research will be completed under the supervision of Prof. Damien Hicks and Prof. David Liley of Swinburne University of Technology.

\section{Conclusions}


\cite{zart}

\nocite{*}
\bibliographystyle{plain}
\bibliography{test}




\end{document}
